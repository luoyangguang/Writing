\section{Experimentation} \label{sec:Experimentation}

In this chapter the results from each type of assessment are presented.
The first assessment is range,
	followed by response time,
	after that connection speed,
	and finally the power consumption.
The only assessment that is not performed on the prototype is the range assessment.

\subsection{Range}
Range is very hard to measure without advanced equipment and isolated
rooms but can be roughly estimated with equation 4.1 called Friis range
equation [37]. P t is the sender transmit power, P r the receiver sensitivity, d
is the distance between the antennas in meters, f is the signal frequency in
hertz, and λ is the wavelength. G t and G r is the antenna gain for the trans-
mitter and the receiver. The last term in equation 4.1, when inverted, is the
Free-space path loss (FSPL) and can be expanded as shown in equation 4.3.
P r (dB) = P t + G t + G r + 20 log 10 (
λ
)
4πd
c
f
FSPL(dB) = 20 log 10 (d) + 20 log 10 (f ) − 147.56
λ =
(4.1)
(4.2)
(4.3)
Unlike Friis range equation, the Link budget equation 4.4 also takes external
loss like FM into account [38]. This is needed to make a correct estimation of
the actual range as there are several things in the environment that obstructs
and distorts the signal.
P r = P t + G t + G r − FM − FSPL
(4.4)
Combining equation 4.1, 4.3 and 4.4 gives us the equation for the estimated
distance as seen in equation 4.5.
d = 10 x
P t + G t + G r − P r − F M + 147.56 − 20 log 10 (f )


With this equation an estimation of the transceiver range can be made for different FMs and transmit powers.
When deployed,
	the transceiver is configured to only accept packages with a signal strength of -70dBm and above to minimize packet loss and corruption.
The antenna gain for OpenMote is 0dBi and can thus be omitted.
Figure 4.1 shows a comparison between three different levels of FM: 0dB, 10dB,
	and 20dB.
A FM of 0dB means that there is no signal loss except the FSPL and this is very hard to achieve outside of a lab environment.
When increasing the FM to 10dB,
	which corresponds to a normal home environment,
	the maximum range drops to 22m.
However,
	in these kind of environments the desired range is usually around 10m which would let the device reduce the transmit power to around 0dBm.
Finally,
	the FM is increased to 20dB which is roughly what it would be in a office or industrial environment.
The maximum range in this environment is now reduced to only 7m when transmitting at maximum power.

Response time
Before measuring the response time, some theoretical estimations are needed
to be able to evaluate the real values. The theoretical values are based upon
the radio duty cycle (RDC) and the average response time to reach a node
can thus be derived from equation 4.6, 4.8, 4.9 and 4.10. As each node only

\subsection{Response time}
Before measuring the response time,
	some theoretical estimations are needed to be able to evaluate the real values.
The theoretical values are based upon the radio duty cycle (RDC) and the average response time to reach a node can thus be derived from equation 4.6, 4.8, 4.9 and 4.10.
As each node only checks the radio every 125ms,
	this duration combined with the data packet send time of ∼4ms (equation 4.7.) and ACK send time corresponds to the worst case delivery,
	as the node needs to wait a whole cycle before being able to send the package the desired node.
When the target node is already listening,
	the best case delivery time is 5ms.
Thus,
	the average theoretical delivery time to reach any adjacent node is 67.5ms.
Radio duty cycle:
	Transfer time: 1s = 0.125s = 125ms 8 (4.6) 133B + 4B ∼ 4ms 31.25KB/s (4.7) Worst case delivery: 125ms + 4ms + 1ms (4.8) Best case delivery: 4ms + 1ms (4.9) 130ms + 5ms = 67.5ms (4.10) 2 The delivery time is only calculating the time to send a packet over a link,
	but when calculating the response time,
	the acknowledge (ACK) response has to be included in the calculation.
Each ACK also needs to wait for the target node to be awake,
	adding one more instance of average delivery time,
	resulting in 125ms in average response time.
This time will multiply with each hop,
	resulting in equation 4.11, 4.12 and 4.13.
Avg.
delivery:
	Avg.
response time:
	(2 · 67.5ms) · hops (4.11) Best case response time:
	(2 · 5ms) · hops (4.12) Worst case response time:
	(2 · 130ms) · hops (4.13) After doing a test with real nodes set-up with a 8Hz RDC with three hops,
	as seen in figure 4.2,
	the values in table 4.1 were obtained.
Each node was pinged 200 times at a one minute interval to simulate some traffic on the network.
What can clearly be seen in the average field of the table is that the average of 765ms is much higher than the expected average of 135ms;
	the difference is mainly due to the worst-case pings that in some cases had response times up to 30 seconds.
However,
	when looking at the geometrical mean which is better at smoothing out big spikes seen in figure 4.3,
	the observed response time is still 265ms which is a bit longer than the expected worst case response time for one hop.
Also,
	for two and three hops the observed average is high,
	but the geometrical mean shows that this is due to the spikes.
The estimated response time for two hops is 270ms which as seen in the geometrical mean table 4.1 is way off by ∼500ms.
The same observation goes for three hops where the observed geometrical mean response time is 1181ms which compared to the estimated response time of 405ms is significantly higher.

With these observations in mind,
	the estimation could be described much better with equation 4.14.
which would result in an average response time of 266, 532 and 1064 ms for one,
	two and tree hops.
However,
	this would mean that the response time is doubled for one hop and then doubled for each consequent hop making the response time exponential which should not be the case.

With the RDC disabled,
	i.e.
the transceiver is always listening and the MCU does not go into sleep mode,
	the response time is completely different.
As seen in table 4.2 the average response time is around 12ms per hop and the spikes seen in the response time for 8Hz RDC is gone.
Furthermore,
	the estimated best case response time of 10ms is very close to the observed average response time.
The response time also scales to the number of hops as expected and is roughly 12ms per hop.
It appears there might be a problem in 


\subsection{Connection speed}
Connection speeds can be measured in several ways each with their own different pros and cons.
One of the most popular ways is throughput,
	i.e.
the amount of data over the link is divided by the time it took to reach the target.
However,
	this gives a false picture of how fast the connection actually is from the developer’s point of view,
	as the measured data does not only contain application data but also headers and checksums.
IEEE 802.15.4 has a theoretical data rate of 250kb/s as seen in equation 4.15.
but this is only a measure of how many bits per second the transceiver is able to output.
The application data part,
	when using no header compression,
	is only 41\% of the total transfer.
Thus,
	resulting in a theoretical application
data rate,
	also called goodput,
	of only 12.81KB/s.
Data rate: 250kb/s = 31.25KB/s 133B − 54B = 0.59 133B Theoretical goodput: 31.25KB/s · (1 − 0.59) = 12.81KB/s Overhead:
	(4.15) (4.16) (4.17) When using CoAP as the application level protocol,
	each package can carry either 32 or 64 bytes of application data.
In practice,
	the 64B mode is only applicable when sending packages between nodes on the same mesh network,
	as the addressing fields then can be fully compressed.
When using applications outside the mesh network,
	each package can only carry 32B of data,
	resulting in a packet size of 111B as shown in equation 4.18;
	this does not affect the theoretical data rate but has a noticeable impact on the goodput due to the large overhead of 71%,
	as shown in equation4.19.
To be able to use the full data rate,
	the application needs to use a protocol without handshakes,
	i.e.
UDP,
	as the transceiver then can send the packets as fast as physically possible.
CoAP is implemented on top of UDP and thus has a low transport layer overhead,
	but uses its own mechanism for handshaking,
	delivery and ordering.
The theoretical CoAP application throughput can be estimated by looking at the average response time of the node and then add that time to the the data delivery time.
Each package needs to be acknowledged before the next package is sent,
	and thus the node response time needs to be taken into consideration.
When doing so,
	the throughput as calculated in equation 4.20.
is only 1.64KB/s and thus the theoretical goodput is reduced from 12.81KB/s down to 0.48KB/s as shown in equation 4.21.
Packet size: 133B − 54B + 32B = 111B Actual overhead: 79B = 0.71 111B (4.18) (4.19) 

Using the packet size of 111B together with the theoretical response time from equation 4.11 would give the results shown in figure 4.4.
To verify these calculations,
	the same test set-up as shown in figure 4.2,
	which also was used in section 4.2,
	was used to test throughput and goodput at different number of hops.
Each node was sent 1KB data each minute for 200 minutes;
	the time from the first package sent to the final acknowledge packet received was measured for each 1KB transmission.
The first test was performed with an RDC of 8Hz and resulted in the values shown in figure 4.5.
As the chart shows,
	the theoretical throughput and goodput is much higher than the observed values,
	but this is due to the fact that the actual average response time is higher than the theoretical one.
With some calculations made the observed throughput and goodput are within range of what is expected,
	given the observed response times in table 4.1.
Equation 4.22 uses the observed values to calculate the average response time,
	given the values in figure 4.5.

\subsection{Power consumption}

To measure power on devices that use very low power and also changes the power consumption very rapidly and frequently is not an easy task.
According to the currency specification from the CC2538,
	the different power modes have the consumption seen in table 4.3 using the built in voltage regulator TSP6750 that switches the input voltage down from the 3V to 2.2V.
The components on the OpenBattery supplied directly by the 3V batteries have the current and power consumption specifications as seen in table 4.4.


Given these power profiles,
	combined with the time it takes to receive and transmit packages,
	and retrieve a measurement,
	the theoretical power for one RDC cycle results in the chart seen in figure 4.7.
The node starts in sleep mode using 112.81μW and after 109ms wakes up and goes into RX mode where a request for a sensor value is received.
The node then switches off the radio and fetches the sensor value.
After the value is retrieved from the sensor,
	the radio is once again put in to RX mode for a Channel Clear Assessment (CCA) before entering TX mode and sending the payload.
The transmission is successful and the node goes into RX mode to listen for the ACK,
	when it is received the node enters sleep mode again.
For this cycle the average power consumption is 4.8mW which would drain the 2250mWh batteries in 19 days.

However,
	as the nodes have a RDC running at 8Mz,
	most of the time there will be no package for the node to receive and thus no measuring and transmitting,
	as seen in figure 4.8.
This cycling reduces the average power consumption to 0.47mW,
	which would make the batteries last for ca 200 days.
The goal is to have a node that can run for one year without having to change the batteries and to be able to do this on 2xAAA batteries with 750mAh the average consumption has to be under 257μW as calculated in 

To verify these assumptions,
	we used a Keithley 2280S power supply [39] to measure the total current draw of the prototype.
The node was connected to the power supply,
	which was set up to make 277 measures each second with a supply voltage of 3V.
Several measurements were performed.
One of the most interesting ones can be seen in figure 4.9.
In this picture,
	we can clearly see the different operating modes,
	as the node performs 3 transmissions during the interval.
In the first transmission at the 1.6s mark,
	the strobing feature of the RDC protocol is seen as the package is sent 5 times before the receiving node is awake and can receive the package.
In the two following transmissions,
	the package is delivered on the first try.
As our measurement is limited to 277Hz,
	the current peaks when only waking up to listen for traffic are sometimes missed,
	and the peak value is hard to extract;
	but the 8Hz RDC cycle is still visible.
The average power consumption for these cycles is 8mW,
	which would make the batteries only last for 11 days.
However,
	when taking the average of a measuring series without any transmissions,
	the average goes down to 4mW,
	which increases the battery time to 23 days.
The theoretical sleep power of 0.11mW compared to the measured of 3mW is what makes the average power consumption that high.

Reducing this power consumption by a tenfold would result in an average consumption of 0.39mW,
	which is closer to the theoretical average power consumption.
A discovery made when measuring the power was that the nodes consumed less power when supplied with a lower input voltage.
Simply by reducing the voltage from 3V to 2.6V reduced the power consumption in LPM by 15\%.
However,
	this reduction could affect the range of the nodes.


Internet of Things can be realised in several ways as there are still many viable options on the market,
	mainly in terms of hardware,
	operating systems,
	and communication standards.
Given the recent development in the field,
	Thingsquare recently released a technology demo using the same practices as used in this thesis;
	the choices taken are on track with the latest development [40].
Also,
	both Google and Microsoft have announced that they are developing IoT OSs.
When these products are released,
	it would be very interesting to compare them with Contiki.
It would be exciting to see if an open-source project can surpass the commercial offerings in terms of speed,
	RAM and ROM footprint,
	and device support.
Furthermore,
	an in-depth comparison between RIOT and Contiki would give much insight into the kind of OS practices that benefit IoT development the most.
Google have also started to develop a substitute for 6LoWPAN and UDP that they have named Thread [41].
As 6LoWPAN and ZigBee,
	it runs on top of IEEE 802.15.4 and thus might be able to out-compete the existing implementations.
Google promises lower latencies and power consumption compared to the existing technologies.

\subsection*{The prototype}

The prototype development took more time than initially planned;
	mostly because of the complexity of the OS,
	but also due to bugs in the untested drivers.
The prototype combines the technology from each field,
	i.e.
hardware,
	OS,
	and communication protocol,
	and fulfils the requirements set in section 3.1.1.
Even though the OS is relatively simple,
	compared to Linux,
	Windows,
	and OS X,
	understanding the mechanics of the RDC driver and the LPM driver was difficult,
	but necessary to be able to interpret the test results.
The prototype worked very well during most of the testing,
	with only a few unforeseen deviations.
One occurred during the power measurement,
	where the power consumption in low power mode tripled in one of the test series;
	this behaviour could not be reproduced and is therefore not included in the results.
Also,
	in the early stages when working with the 8Hz RDC driver,
	packet losses over 50\% were recorded for packets with more
than one hop;
	this problem was solved,
	when a new version of radio driver was released by the OS development team.
Selecting OpenMote to be the hardware platform together with Contiki as the OS,
	was a very good choice as companies are starting to build their IoT solutions around Contiki and similar hardware platforms [42, 43].
Already in the beginning of the development,
	several benefits were noticed;
	new drivers and bug-fixes were released increasing the stability and functionality of the OS.
The active community around the combination of OpenMote and Contiki was really helpful when developing the drivers for the I 2 C and sensor drivers.
Example projects for other platforms could be used as references,
	giving much insight to how the programming for this type of OS worked.
It would have been interesting to examine the differences between two operating systems;
	not only to test which one has the better performance,
	but also to compare which one that has the more favourable code structure and development procedure.







