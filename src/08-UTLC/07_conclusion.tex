\section{Conclusion} \label{Sec:Conclusion}

% Restate the main challenges
In this paper,
	we proposed an Urban Traffic Light Control (IoT-UTLC),
	considering its architecture’s elements and tools used to build an IoT lockup.
We reported three main contributions in this work:
	i) Modelling through UPPAAL of crossing's traffic lights,
	ii) Prototyping of IoT Edge Computing,
	iii) Performance assessment of MQTT protocol.
Traffic lights control has been taken as an IoT network.
WSN has been deployed on motes running Contiki Os and exchanging IEEE802.15.4/6LowPAN packets.
MQTT was the QoS protocol between our WSN and Ubidots Cloud platform.
UPPAAL model checker design ensured that lights' colors change is adaptive to the arriving of a priority vehicle.

% Restate the main findings 
Our experiments have investigated the relationship between the MQTT protocol and the traffic flow congestion.
Our results showed that the MQTT is efficient when the number of packets sent exceeds 35\% of the total number.
The packets with the highest level of QoS has low latency than other packets.
The protocol remains efficient since the delay of priority packets decreases when the network overhead increases.
However,
	found latencies of up to 400ms is higher than the expected one for vehicular safety application.
Thus,
	the proposed IoT architecture and protocols must be improved to consider the safety requirements.

% Future challenges current bad state %
While we are still developing our prototyping,
	we plane to integrate other use cases such as smart buildings and industrial IoT.
% Improvements
As a near future work,
	we plane to extend our experiments with private Cloud towards a real Fog Computing.

\section*{Acknowledgement}

We would like to thank our colleague Sebti Mouelhi,
	associate professor at ECE Paris,
	who provided us insight and expertise on UPPAAL.
