Objectifs de votre plan d'affaires : 
    • Présenter le projet
    • Démontrer la faisabilité et la validité
    • Convaincre
    • Fournir les éléments-clés et les éléments qui seront communiqués au comité d'engagement

\section{D'abord, une page résumé :}
	\subsection{marché}
	\subsection{produit / service}
	\subsection{l'équipe}
	\subsection{avantages de l'entreprise}
	\subsection{objectifs visés}
	\subsection{l'estimation et de la nature des besoins financiers}
	\subsection{montant des fonds recherchés}
	\subsection{la rentabilité visée}
	\subsection{temps de retour sur capital investi}
	\subsection{l'image visée pour l'entreprise}

\section{Présentation du projet :}
	\subsection{Historique}
	\subsection{Famille de produits / services}
	\subsection{Le projet de développement : innovation, croissance externe}
	\subsection{Création de l'entreprise : date, nom, forme juridique (pourquoi), brevet  / licences}

\section{Le marché :}
	\subsection{Concurrence :}
		\subsubsection{Laquelle ?}
		\subsubsection{Quelle place pour une entreprise différente ?}
	\subsection{Études de marché :}
		\subsubsection{Lesquelles ? Quelles sources ?}
		\subsubsection{Quelle confiance leur accorder ?}
		\subsubsection{Sont-elles adaptées aux produits nouveaux ?}

\section{Le produit / service :}
	\subsection{Physique}
	\subsection{Fonctionnelle}
	\subsection{Technologies}
	\subsection{Avantages par rapport à la concurrence}
	\subsection{Maîtrise des inconvénients / risques}

\section{Propriété intellectuelle de l'innovation :}
	\subsection{brevet (s)}
	\subsection{licences exclusives ou non}
	\subsection{savoir-faire}
	\subsection{qui gère et contrôle la RID}
	\subsection{annuités, plan de RID, accès à l'évolution}
	\subsection{veille : équipe interne / externe}

\section{Les objectifs :}
	\subsection{Coûts (investissements / exploitation)}
	\subsection{Délais de conception / fabrication / durée de vie}
	\subsection{Performances  :}
		\subsubsection{Parts de marché et fidélisation clientèle}
		\subsubsection{Performances du produit / service}
		\subsubsection{Rentabilité / ROI / Dividendes}
		\subsubsection{RID, veille,}
		\subsubsection{Intéressement des salariés}
		\subsubsection{Communication}

\section{Les risques :}
	\subsection{Analyse des risques internes et externes au projet}
	\subsection{Moyens de les anticiper et de les maîtriser}
	\subsection{Avantages du projet}
	\subsection{Inconvénients du projet}
	\subsection{Moyens mis en œuvre pour les assurer}
N.B. : Ne masquer aucun risque ! Montrez comment votre équipe va le maîtriser

A partir des risques externes, un Comité de Pilotage :

…. Et à partir des risques externes, un Chef de projet, puis une équipe de projet 

\section{L'équipe :}
	\subsection{Les hommes-clé}
	\subsection{L'équipe}
	\subsection{La motivation (intéressement, stock options, …)}
	\subsection{L'état d'esprit visé (gagnant-gagnant, transparence)}
	\subsection{La communication interne}
	\subsection{L'organisation interne}
	\subsection{Dynamiser créativité et innovation}

\section{Le plan de RID :}
	\subsection{Les hommes-clés}
	\subsection{Le plan}
	\subsection{L'organisation}
	\subsection{La veille}
	\subsection{Les coopérations et la gestion du changement}

\section{Le plan marketing :}
	\subsection{Parts de marché et segmentation}
	\subsection{Distribution (Europe, régionale, locale, hors UE)}
	\subsection{Politique de prix}
	\subsection{Administration des ventes}
	\subsection{Communication : publicité, promotion, média, site Internet, réseau-relais,}
	\subsection{Budgets commerciaux}

\section{Le plan d'industrialisation :}
	\subsection{Qui fabrique et comment ?}
	\subsection{Immobilisations et investissements}
	\subsection{Sous-traitance, fournisseurs et esprit gagnant-gagnant}
	\subsection{Sécurisation des savoir-faire}
	\subsection{Logistique}
	\subsection{Qualité et contrôle}

\section{La stratégie résumée :}
	\subsection{Pourquoi la stratégie est la meilleure vis à vis des objectifs et contraintes ?}
	\subsection{Conséquence en termes :}
		\subsubsection{de structure juridique}
		\subsubsection{de mode de gestion}
		\subsubsection{de jalonnement et de contrôle}

\section{Le contrôle :}
	\subsection{Mettre en place un contrôle "à zoom", ni lourd ni pesant ni inadapté ...}
	\subsection{Pas de paralysie de l’équipe de projet}
	\subsection{Pas de déresponsabilisation}
	\subsection{Un tableau de bord simple}
N.B. : Tous les membres de l'entreprise sont co-responsables du contrôle, en continu …

\section{Le montage financier et juridique :}
	\subsection{Décomposition analytique des ventes à 5 ans}
	\subsection{Décomposition analytique effectifs et salaires / an}
	\subsection{Charges d'exploitation / an}
	\subsection{Immobilisations et investissements}
	\subsection{Comptes de résultats prévisionnels}
	\subsection{Dont Plan de financement}
	\subsection{Et dont Bilans prévisionnels}
	\subsection{Plan de trésorerie mensuel à 1 an}
D'où les besoins en fonds propres et la valorisation de l'entreprise











































\section{Marché visé}
\section{Produit}
\section{Equpe}
\section{Avantage de l'entreprise}
\section{Objectifs}
\section{Estimation da la nature des besoins financiers}
\section{Montant des fonds recherchés}
\section{Temps de retour sur le capital investi}
\section{Image pour l'entreprise}
\section{Objectifs quantifiers}
	\subsection{Cout}
		\subsubsection{Investissement}
			\paragraph{Phase de mise en place}
			\paragraph{Phase de conception}
			\paragraph{Phase d'évolution}
		\subsubsection{Maintenance}
			\paragraph{Phase d'exploitation}
			\paragraph{Phase de démentalement}
	\subsection{Délais}
		\subsubsection{Lancement}
			\paragraph{Phase de mise en place}
			\paragraph{Phase de conception}
			\paragraph{Phase d'évolution}
		\subsubsection{Durée de vie}
			\paragraph{Phase d'exploitation}
			\paragraph{Phase de démentalement}
	\subsection{Performance}
		\subsubsection{segmentation du marché -> Cahier des charges}	
\section{Compte prévisionel}


