\section{Introduction} \label{sec:Introduction}

Modern and even old cities have several issues regarding building management,
	urban mobility,
	disaster recovery,
	which can be supported by \ac{IoT} applications.
Actually,
	different kinds of services appear to help city management by automating tasks and predicting critical situations before they occur.
\ac{EES} for example could help to locate survivors and plans for rescue missions in case of disasters.

\ac{EES} require high quality level of communication to ensure their proper operations in critical situations.
This constraint is more complex when devices need to communicate in a large area like building sites.
Thus,
	the need for a powerful network technology that could transmit data with a long range and sufficient \ac{DR} is very challenging.

First the architecture that we propose are based on three component,
	\ac{IoT} devices,
	smart gateway and edge computing server.
\ac{IoT} applications,
	especially emergency ones in case of disasters,
	require data to be received and bounded in time intervals.
So,
	service orchestrations is an important rigid constraint.
Hence,
	the orchestration algorithms should adapt the network configuration to the emergency situation of the site.
Furthermore,
	the delay to adapt network behavior to the emergency situation has to be optimized to react quickly when an alarm is triggered.
For this reason,
	we opt for the use of an edge computing and a smart gateway which could map transmission tuning to the emergency state.

% % Needs Requirements Constraints by Statistics (2-3 lines)------------------------------------------------
% According to 
% The need of
% The main factor is

% % Difficulties ans Challenges (1-2 lines) ------------------------------------------------
% The difficulty to build such system is 

% % Problem Statement (4-5 lines) ------------------------------------------------

% % Goal (3-4 lines) ------------------------------------------------

% % Clear Contribution (3-4 lines) ------------------------------------------------
% For that purpose,
% we ...
% In this work we ...

% The structure  ------------------------------------------------------------------------
This paper is organized as follows.
\refsec{Related work} elucidates summary of related works.
\refsec{Background} highlights the required network technologies that could be used in \ac{EES}.
In \refsec{Approach},
	we propose our architecture scheme for \ac{EES}.
% Our experimentation is presented in \refsec{Experimentation}.
% Our findings are presented in section \ref{sec:Results}.
\refsec{Conclusion} concludes this paper.

