\section{Conclusion} \label{sec:Conclusion}

% % Restate the main challenges ------------------------------------------------
% The main challenge of this work was to
% The efficiency of such algorithms

% % Restate the main contribution ------------------------------------------------
% Our main contribution was to

% % Restate the main findings --------------------------------------------------
% To measure the accuracy of

% % Future challenges current bad state -------------------------------------------
% As we find that ..., we plan to  ..

\ac{LPWAN},
	\ac{WSN} and \ac{IoT} architecture are the first candidates to enhanced disaster monitoring and management systems.
Particularly,
	IEEE802.15.4 and LoRa networks gives new insight for effective \ac{EES}.
This work gives an overview of deployment of \ac{IoT} architecture for \ac{EES},
	Such services and demand for edge computing in real-time poses new architectural and service orchestration challenges.
As a future work,
	we plan to study the efficiency of using artificial intelligence/machine learning to adapt these two networks to the emergency situation of the building.
% It is concluded that future research should focus on developing artificial intelligence/machine learning driven more robust radio resource management strategies to enable optimized operations in real-time.

% In this chapter,
% limitations of two major IoT standards (LoRA and 4G LTE) are presented with some future research recommendations.


