% Affichage des résultats Graphiques, tableaux
% Interprétation des résultats: Les tendances, les cas extrêmes, les cas favorables, les cas défavorables
% Comparaison avec les résultats des contributions existantes par rapport aux hypothèses considérées
% Justifier le choix de l'approche avec des résultats
\subsection{Results}

\begin{frame}[fragile]{Results}{Comparison}

	\Columns{0.5}{0.5}{

		Initial values:
		\Itemize{
			\item generated randomly (normal distribution)
			\item represent individual vulnerabilities.
			\item dark color = highly infected
		}
		Final values:
		\Itemize{
			\item obtained after convergence.
			\item represent social vulnerabilities.
		}
	}{
	\FigureH{h}{0.45}{local_.png}  {Individual privacy vulnerability}
					{social_.png} {Social privacy vulnerability}
					{graph}       {Individual \& Social privacy vulnerabilities}

		\Table{c|c|c}{table1}{Individual and social privacy vulnerabilities}{
			\bf User ID & \bf Individual Vul & \bf Social Vul \\\hline
			34          & 0.84               & 0.67           \\
			67          & 0.12               & 0.87           \\
			206         & 0.76               & 0.33           \\
			588         & 0.23               & 0.78           \\\hline
		}
	}
\end{frame}


\begin{frame}{Results exploitation}

	\TickzH{h}{.45}{figure1}{Enron dataset}{figure3}{Caliopen dataset}{cdf_graph}{Cumulative distribution function of infected users}{.5}

	\Itemize{
		\item Figures shows the CDF of the vulnerability diffusion process.
		\item The vulnerability diffusion process increases as the reputation level of vulnerable users increases.
		\item Users with high reputation values contribute significantly to the diffusion
		\Itemize{
			\item They spread their vulnerabilities quickly and widely through the network.
		}
	}

%	\begin{columns}
%	
%		\begin{column}{0.5\textwidth}
%			\Itemize{
%				\item Figures shows the CDF of the vulnerability spreading.
%				\item the vulnerability diffusion process increases as the reputation level of vulnerable users increases.
%				\item Users with high reputation values contribute significantly to the diffusion
%				\item They spread their vulnerabilities quickly and widely in emaillll.
%			}
%		\end{column}
%		
%		\begin{column}{0.5\textwidth}
%			\TowTickzV{h}{.45}{figure1}{Enron dataset}{figure3}{Caliopen dataset}{cdf_graph}{CDF of infected users}
%		\end{column}
%	\end{columns}
	
\end{frame}

\begin{frame}{Results exploitation}

	\TickzH{h}{.45}{figure2}{Enron dataset}{figure4}{Caliopen dataset}{cdf_graph}{Convergence of the diffusion process}{.5}
	\Itemize{
		\item The process converge when the mean distance between social vulnerability scores is the minimum.
		\item Assigning trust to vulnerable users allows them to achieve a high level of reputation.
		\item Consequently, they infect all other vulnerability values.
	}
			
			
%	\begin{columns}
%		\begin{column}{0.5\textwidth}
%			\Itemize{
%				\item Assigning trust to vulnerable users allows them to achieve a high level of reputation.
%				\item Consequently, they infect all other vulnerability values.
%				\item The process converge when the mean distance between social vulnerability scores still the same.
%			}
%		\end{column}
%		\begin{column}{0.5\textwidth}
%			\TowTickzH{h}{.45}{figure2}{Enron dataset}{figure4}{Caliopen dataset}{cdf_graph}{Convergence of the diffusion process}{.5}
%		\end{column}
%	\end{columns}

\end{frame}


